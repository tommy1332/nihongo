\documentclass[a4paper,11pt]{scrbook}

\usepackage{xeCJK}
\usepackage{ruby}
\usepackage{amsmath}
\usepackage{geometry}
\usepackage[ngerman]{babel}

\setCJKmainfont{DFKyoKaSho-W4}

\newlength{\kanjispace}
\setlength{\kanjispace}{0.8cm}

\parindent 0pt

\newcommand{\utext}[2]{$\underset{\mbox{\tiny #1}}{\underline{\hspace{#2\kanjispace}}}$}
\newcommand{\uline}[1]{\underline{\hspace{#1\kanjispace}}}
\renewcommand\rubysep{-0.1em}

\title{\ruby{日本語}{にほんご}を\ruby{勉強}{べんきょう}しましょう!}
\subtitle{\emph{nihongo o benkyō shimashō!}\\Lernen wir Japanisch!}
\date{}
\author{田中ねこ\\山田いぬ}

\begin{document}
  \newgeometry{centering}
  \begin{titlepage}
    \maketitle
  \end{titlepage}
  \restoregeometry

Diese Seite ist leer.
\tableofcontents

\chapter{Vorwort}
\begin{itemize}
  \item Intendierte Zielgruppe:
    \begin{itemize}
      \item Gute Deutschkenntnisse
      \item Keine Vorkenntnisse im Japanischen nötig
      \item Kenntnisse der Aussagenlogik und Cantor'schen Mengentheorie
    \end{itemize}
  \item Ziele:
    \begin{itemize}
      \item Vorbereitung auf N5
      \item Lehrbuch
      \item Vermittlung von grundlegenden Kenntnissen im Lesen und Schreiben (Sprechen, Hören)
      \item Vermittlung von Wissen über japanische Kultur
    \end{itemize}
\end{itemize}

\end{document}
